%************************************************
\chapter{Background on Event-Driven Business Process Management}\label{ch:background}
%************************************************

\section{Business Process Management}\label{ch:bg:bpm}
With its origins dating back to the process orientation trend of the 1990s, \ac{BPM} has meanwhile become a mainstream tool to support organizations.
It had been noted, that company workflows can essentially be broken down into activities that are executed in a coordinated manner by one or more parties.
A certain group of activities thereby form a process which is executed within an organization.
More precisely, \citeauthor{weske:bpm-book} \cite{weske:bpm-book} defines a single \emph{business process} as follows:

\begin{description}
	\item[Definition 1 (Business Process):]
	A \emph{business process} consists of a set of activities that are performed in coordination to realize a business goal. Each business process is enacted by a single organization, but it may interact with processes performed by other organizations.
\end{description}

\noindent The term \emph{Business process management} describes the techniques available to develop and support processes throughout their life-cycle.
It is grounded in the use of explicit process representations which ultimately allow the exchange, analysis and reproduction of the workflows.
This process specification is referred to as the \emph{business process model}, composed mainly of activities and the rules that are necessary to coordinate their execution.
When a process is performed according to its model, the single execution is called \textit{process instance}.
Based on a process model, the number of possible instances is theoretically unbounded.

\begin{description}
	\item[Definition 2 (Business Process Model):]
	A \emph{business process model} consists of a set of activity models and execution constraints between them. A \emph{business process instance} represents a concrete case in the operational business of a company, consisting of activity instances.
	\cite[p.~7]{weske:bpm-book}
\end{description}

\noindent
The life-cycle of a business process can be described by four phases in that numerous stakeholders interact and contribute depending on their specialization.
Process development starts with a \emph{Design~\&~Analysis} phase which yields a refined and validated business process model.
In the following \emph{Configuration} phase, it is necessary to prepare the process implementation, select the means and an environment to run the process in.
The action of making the process runnable in the execution environment is called \emph{process deployment}.
After these preparations the process can be enacted in daily business while its current state is monitored and system maintenance is performed if necessary (\emph{Enactment Phase}). 
A single process execution begins with the \emph{process instantiation}, when the process has succeeded or is aborted, we say the process is \emph{terminated}.
During the enactment, system and stakeholders can start collecting performance indicators and process execution logs to allow evaluating the quality of the process specification. If that \emph{Evaluation} step reveals deficiencies, the life-cycle starts over by entering the design phase once again.
The \emph{process un-deployment} is performed if necessary, so that no new instances of the old process can be started~\cite[p.~11~ff.]{weske:bpm-book}. A similar life-cycle is described by Dumas in \cite{dumas:bpm}.
%\todo[inline]{add reference to \cite{dumas:bpm}, mention that Weske says pretty much the same}
%the investigations undertaken in this work target the Design \& Analysis phase

While traditionally activities are executed manually by company staff following the written process specifications, computer systems are used today to drive the execution and enforcement of business processes and organizational rules.
The generic software systems utilized for that purpose are introduced in the following section.

\subsection{Business Process Management Systems}\label{ch:bg:bpms}
The implementation of business processes has developed from a manual execution guided by business rules to a fully automatized execution in a specialized IT environment.
One of the main reasons for \acs{BPM}'s growing popularity is that in today's fast-paced economy, a large part of the business activity is either supported by computers or even carried out autonomously by them.
The specialized software systems that are utilized to support the enactment of business processes are referred to as \emph{business process management systems}.

\paragraph{Process Management Architecture}
A typical IT infrastructure for driving business processes is illustrated in \autoref{fig:bpm-architecture}. 
Five principal building blocks are considered which will be explained in the following.

With reference to the business process lifecycle, the visualized scenario commences with the \emph{Business Process Modeling}. As a result of the \emph{Design \& Analysis} phase, new process models are created and refined to be stored in the \emph{Business Process Model Repository}. 
The relation between the two elements includes writing new models to the repository as well as reading models for review and further modification.
Given that the desired model is approved for enactment, the process gets deployed to the \emph{Process Engine} as part of the configuration step.
The process engine is the heart of the execution environment. It performs the execution of the processes from deployment until un-deployment, while the enactment and instantiation is controlled by the \emph{Business Process Environment}.
An indefinite number of \emph{Service Provider}s realize application services to support the process execution. A service provider can be a software module but also a knowledge worker performing a particular process step.

\begin{figure}[]
	\myfloatalign
	{\includegraphics[width=1\linewidth]{chapters/background/bpm-architecture.png}}
	\caption{Business process management systems architecture model (see~\cite{weske:bpm-book},~p.~120)}
	\label{fig:bpm-architecture}
\end{figure}


\paragraph{The Camunda Business Process Engine}
A large, growing number of process engines is available on the market, including solutions from IT giants like SAP, IBM and Oracle.
In this work, \emph{Camunda BPM}~\cite{camunda} has been chosen to illustrate implementations.
As of August 2017, the software product is available in version 7.7.0 and comes in a commercial, regularly updated version and in a free, community-driven solution that is updated with every major release.
Camunda is popular among the research community as the source code is openly available, the product is mature, but actively developed and offers comprehensive support for BPMN 2.0. It is designed to be extensible and easily modifiable to adapt to custom requirements.
\emph{Camunda BPM} comprises a modeling tool, the Camunda process engine core and a number of browser-based user-interfaces to control process enactment and monitor execution state.
\autoref{ch:implementation} will provide further details about the engine architecture and extension mechanisms.


\subsection{Business Process Model and Notation}

A generic meta model of business process models is proposed in \cite{weske:bpm-book}.
According to the model, all processes are composed of nodes and edges with each node representing either an activity model, an event model or a gateway model. The complete definition is provided in \textit{Definition~3}.

\begin{description}
	\item[Definition 3 (Business Process Meta Model):]
	Let $C$ be a set of control flow constructs. $P = (N,E,type)$ is a \textit{process model} if it consists of a set $N$ of nodes, and a set $E$ of edges. \cite{weske:bpm-book},~p.~91
	\begin{itemize} 
		\item
		$N = N_{A}\cup N_{E}\cup N_{G}$, where $N_{A}$ is a set of activity models, $N_{E}$ is a set of event models and $N_{G}$ is a set of gateway models. These sets are mutually disjoint.
		\item 
		$E$ is a set of directed edges between nodes, such that $E\subseteq N \times N$, representing control flow.
		\item
		$type:N_{G}\rightarrow C$ assigns to each gateway model a control flow construct.
	\end{itemize}
\end{description}

Given the general semantics of business processes, a specific modeling notation has to be selected to express an informal process description in a formal, interchangeable way.
Different languages and notations have become available over the years, each serving different specializations.
Kossak~et~al.~\cite{kossak:bpmn2} organize some of the more popular languages as follows: A subset of them are focused on the control flow of business processes, for instance \ac{BPMN}~\cite{bpmnspec}, Yet Another Workflow Language and Petri Nets; some focus on object-orientation, like the \ac{UML} activity diagrams and use case diagrams; some are data-flow oriented, e.~g.~the Structured Analysis and Design Technique.
%\todo[inline]{references to the other languages}

Among these, the \acf{BPMN} has developed into a widely-adopted industry standard, also becoming ISO-standard in 2013~\cite{iso2013bpmn}.
The specification is developed by the Object Management Group~\cite{omghome} and now available in version 2.0~(January~2011) after being first released in January~2008. Whenever referring to BPMN in this work, version 2.0 of the standard is meant.
\acs{BPMN} can be understood as an extension to the abstract business process meta model adding a comprehensive catalog of visual representations and semantic constructs on top of a meta model. Furthermore, one of the most important features of its latest version is the a standardized interchange format provided through an \acs{XML} specification, as \cite{weske:bpm-book} points out.
As emphasized by \citeauthor{Muehlen:2007}~\cite{Muehlen:2007}, the increased expressiveness of modern languages like \acs{BPMN} comes at the cost of an increased complexity. An aspect that, apparently, did not stop it from gaining popularity.

\paragraph{Elements of a BPMN Model}
Following the abstract business process meta model, the core elements in any BPMN model are flow elements~(nodes) and connecting objects~(edges).
Flow elements can be either \textit{Events}, \textit{Activities} or \textit{Gateways}, each of them coming in different variations.
This section will introduce a subset of the elements available through the \acs{BPMN} specification to build the foundation to comprehend the thoughts presented in this work.

\begin{figure}[]
	\myfloatalign
	{\includegraphics[width=1\linewidth]{chapters/background/intro-rental-car.png}}
	\caption{Simple BPMN model of issuing a quote for car rental}
	\label{fig:simple-bpmn-model}
\end{figure}

\autoref{fig:simple-bpmn-model} shows how a booking request might be handled in a car rental business.
Circular elements represent events, diamond-shaped elements are gateways. Activities are visualized by rectangles with rounded corners.
The given process gets instantiated whenever a booking request request is received from a customer, shown as a Message Start event. 
As a first step, the employee assigned to handle the request must check if the desired car is available. To that follows an exclusive OR-Gateway, distinguishing the further process flow depending on the availability of the car.
If the car is available, the quote must be created in two sequential tasks to be then sent out by the \textit{Send Message Event}. If the car is not available, the customer is informed about the closing of his request. 
In either case, the process ends with an \textit{End Event} after the customer was informed about the result of his request.

The example illustrates the basic use of activities, events and gateways in BPMN. However, for each type of element, activities, gateways and events, BPMN offers a variety of different kinds.
%Each of them comes in a broad variety.
The Task elements shown in the example~(\eg \,\textit{Check car availability}) are the most basic type of activities, representing \textit{a~unit of work}. The nature of a task can be further specified using task types, all activities can be additionally annotated with activity markers. Each of these modifications describes the activity semantics in a more detailed fashion.
Apart from the utilized XOR-gateway~(the execution-flow will proceed in exactly one of the outgoing branches), there is for example the parallel gateway, activating all outgoing branches. Moreover, the Event-based Gateway, which is followed only by events and continues along the branch of the first event that occurs.
Last but not least, message events are used to represent the action of sending information in the form of a message to a certain recipient. Other available event types are, for instance, timer, signal or error events. Events can occur as \textit{Start Events} (\eg \,\textit{Receive Booking Request}), where they trigger the instantiation of a process. Furthermore as intermediate events (\eg \,\textit{Send Quote}) or as end events, being fired when the process terminates. They can be \textit{catching}, \ie \,receiving a trigger, or \textit{throwing}.

Multiple participants taking part in a process can be expressed using \textit{pools}, which can be sub-divided into \textit{swimlanes}.
Interactions between different pools take place using message flows, which are depicted by a dashed arrow.
An example of such a \textit{Collaboration Diagram} is shown in \autoref{fig:bg:rental-car-customer}. It describes the previously discussed car rental process from the side of the customer, including the information exchange between rental company and customer.
When the airline issues the booking confirmation of the flight, the customer has to book a rental car and a hotel. The hotel booking is modeled using a collapsed sub-process to reduce the complexity of the model.
After issuing the rental car request, an event-based gateway shows that the process is ready to consume any one of the two event messages. If the booking was successful, the control flow proceeds to the inclusive join gateway and the process can terminate once the hotel booking is also completed. If the car rental company informs about unavailability, the customer starts looking for an alternative car option.
A data object is used to model the \textit{Request} artifact as output of the \textit{Prepare~Request} task and as input to the message throw event.

\begin{figure}[]
	\myfloatalign
	{\hspace*{-0.8cm}\includegraphics[width=1.1\linewidth]{chapters/background/intro-rental-car-customer.png}}
	\caption{Rental car booking process with multiple participants}
	\label{fig:bg:rental-car-customer}
\end{figure}


\section{Complex Event Processing}\label{ch:bg:cep}
The IT world is facing an exponential increase in the amount of produced data. A significant part of this data are pieces of information about real-life occurrences, such as a current sensor value, an interaction on a website or the location of a vehicle on the road.
We call this kind of strongly time-related information an \textit{Event} and the according computer science field \acf{CEP}~\cite{evtprocessing}.
More specifically, Etzion and Niblett define an event as \textit{an occurrence within a particular system or domain; it is something that has happened, or is contemplated as having happened in that domain}.
It is understood that events are of a certain \textit{event type}, defining the attributes that each event instance of the event type is composed of. An attribute is described by a unique attribute name and a data type.

\begin{description}
	\item[Definition 4 (Event):]
	An event is a tuple $e = (et, eventtime, c)$, whereas
	\begin{itemize} 
		\item
		$et$ is the event type
		\item 
		$eventtime$ is the time the event happened
		\item
		$c$ is the content of the event consisting of a set of key-value-pairs according to the content description cd of the corresponding event type $et$.
	\end{itemize}
\end{description}

%An event processing network is a collection of event processing agents, producers, consumers, and global state elements, connected by a collection of channels

Four major components take part in an event processing network: (a) An \textit{event producer} provides information to the system, (b) an \textit{event agent} processes the occurring information, so that it can be delivered to the \textit{event consumer}~(c). The components are linked through \textit{event channels}~(d).
Typically, a so-called \textit{\acs{CEP} Engine}~(also: CEP platform) is at the heart of the system, taking the role of an event agent.
Modern CEP platforms are trimmed to maximum efficiency, being able to process hundreds of thousands of events every minute.
Their main purpose is to accept incoming events from event producers, filter and match them according to selection criteria and, finally, derive a new event occurrence to be sent to the registered event consumers.

% cite luckham2008power

\paragraph{The publish/subscribe Principle}
In event-based architectures, communication takes place according to the \textit{Publish/Subscribe Principle}.
The concept essentially demands that an event processing middleware publishes events to processes only after they have issued a subscription for these events.
Consequently, there is a strict temporal order between the actions subscribe, consume and un-subscribe(\autoref{fig:bg:pubsub-workflow-temporal-order}). The consumption and un-subscription can only happen after the subscription. Once an un-subscription has been issued, no consumption can follow.~\cite{tanenbaum:2007}

% mandal also references Barros, A., Decker, G., Grosskopf, A.: Complex events in business processes. In: Business Information Systems. Springer (2007)

\begin{figure}[]
	\myfloatalign
	{\includegraphics[width=0.6\linewidth]{chapters/background/pub-sub-statemachine.png}}
	\caption{Publish-Subscribe workflow expressed in a state diagram}
	\label{fig:bg:pubsub-workflow-temporal-order}
\end{figure}


One of the main advantages of this principle is, that the involved parties are \textit{referentially decoupled}. They do not need to explicitly refer to each other, an aspect that is also acknowledged in~\cite{evtprocessing}.
Their decoupled nature facilitates the management and development of event processing networks. Event producers and consumers might change frequently.
Whenever a new event source is available it can be connected to the CEP platform without considering all future consumers. Consumers can subscribe and un-subscribe without influencing the operations on the consumer side.


\paragraph{Stream Processing}
To be able to cope with potentially large amounts of data, Complex Event Processing Platforms work on the basis of \textit{stream processing}.
In a traditional relational database, information is stored for an indefinite amount of time. When a user queries the data store, the system processes the tabular data and calculates the requested result. 
The advantage of this approach is that the user can access historic data at any time, as long as it is not explicitly deleted from the database.
In many occasions, the amount of available data significantly surpasses the storage capacities and that concept can no longer be followed.

Stream processing addresses the mentioned challenge by largely reducing the amount of data that is persisted in the system. Instead, it is the goal to keep only those pieces of information, that are necessary to process a result for currently registered queries.
Incoming data objects, or events in the case of a CEP platform, enter the event stream and are immediately evaluated against all existing query expressions. If the information is not required to process any of the queries, it is deleted instantly. If an event matches a query, a notification is sent to the subscribed consumers. \autoref{fig:bg:stream-processing} illustrates the concept.
As a consequence, a certain event can not be part of a query result, if that query has been registered after the occurrence of the event.
In case aggregated information is demanded by the query, the stream processor will internally store aggregated information, but not keep every information that led to the aggregated value.~\cite{streamprocessing} 

\begin{figure}[]
	\myfloatalign
	{\includegraphics[width=0.7\linewidth]{chapters/background/cep-stream-processing.png}}
	\caption{Stream processing concept applied in a CEP Engine}
	\label{fig:bg:stream-processing}
\end{figure}

The subscription to an event in a CEP platform is primarily defined by an \textit{Event Query}.
Many modern event query languages are inspired by \acs{SQL}, but cannot be entirely compliant due to the different underlying data processing concept.
When formulating event queries, it is essential to consider the stream processing principle.
For the illustration of the concepts, this thesis relies on the Esper \ac{EPL}~\cite{esperhome}, utilized in Esper-based event processing engines like the one employed in the presented reference implementation, \autoref{ch:implementation}.
A simple event query in Esper looks as follows:

\begin{lstlisting}[language=sql,caption={Sample Query in Esper EPL},label=lst:epl-query-example]
	SELECT occurrencetime, delay, delayreason 
	FROM eurotunnel.win:time(2 hour)
	WHERE delay > 30
\end{lstlisting}

\section{Event-driven Business Process Control}
The disciplines of Complex Event Processing and Business Process Management are connected in \acf{EdBPM}, which discusses the uses of events to enhance business processes~\cite{evtprocessing}.
There are two usage scenarios for events in BPM. In Business Activity Monitoring and analysis, events are published by the process engine and processed to obtain analytical information, for instance about process status, efficiency or errors.
This thesis considers events from the second perspective, \acf{EdBPC}. The field focuses on the business process as an event consumer.
Within business processes, event-based communication can for example be used to exchange information between participants or to react to external events. While the BPMN supports a large variety of different events, we are going to investigate message events specifically. Whenever talking of an event-occurrence, it can be assumed that the term \textit{event} implicitly refers to a message-event in the BPMN context.

\ac{EdBPC} using external event sources is facilitated by connecting the business process engine to an event engine.
As explained in \autoref{ch:bg:cep}, interactions with a CEP platform follow the publish-subscribe paradigm.
In the BPM scenario, this means that a subscription to events must be present, so that events can be received by the process engine and correlated to a specific message element within a process instance~\cite{Baumgrass2016}.
The correlation has to be performed by the process engine on the basis of context information, for example a subscription identifier incorporated inside every CEP message. 
An interaction between process engine and event engine is illustrated in \autoref{fig:bg:subscription-workflow}~\cite[,\,p.\,13]{mandal:2017} at the example of the event engine Unicorn.
Note that a different event engine might implement the work-flow slightly differently, but still follow the same general concept.
In the given case, the process engine first issues a \acs{HTTP} POST call containing the subscription query a notification-path. The path contains the address of the desired notification recipient in case a matching event occurs.
The event engine answers that call with a unique identifier associated to that single subscription.
As soon as an event occurs that matches the query, the event engine sends the query output along with the subscription identifier to the process engine which can correlate the event back to the associated instance.
In the next chapter, we will discuss when exactly the event subscription is issued by the process engine.

\begin{figure}[]
	\myfloatalign
	{\includegraphics[width=0.6\linewidth]{chapters/background/subscription-workflow.png}}
	\caption{Even-subscription work-flow between process engine and event engine.}
	\label{fig:bg:subscription-workflow}
\end{figure}

%- no standard yet available to do subscription in bpmn
%- it must be assumed that the subscription is either already active or explicitly modeled in BPMN, e.g. using a service task
%- OR given the BPMN spec it is generally assumed that the subscription is executed as soon as en event is enabled
%- further analysis of this topic is provided in ...
%- the time of event subscription is clear for start events
%- Correlating events to process instances





