\section{Design Decicions}\label{ch:designdecisions}

The target functionality of the BPMN extension was clearly defined by the identified requirements.
To implement that functionality, there were a number of options to consider and design decisions to make.
This chapter provides background information on the decisions that have influenced the presented concept for flexible event subscription.

\todo[inline]{Can I put together an alternative solution? Table of events for a certain event source. The buffer is already there for me to pick from when designing the process.}

\paragraph{The time of subscription is a question of process design}
-> process designer
\todo[inline]{Is there an additional event engineer?}
- this is why the bpmn extension is presented first
- hide complexity from the designer

\paragraph{The actual buffer is mostly hidden from the user}

-Buffers are implicitly defined through the BPMN model

\paragraph{Buffers are closely linked to process models}
Messages are only buffered as soon as they are explicitly required by a model
- we don't just buffer n messages because we might need them in the future. That would be a fuzzy, incalculable performance overhead.
- instead we keep as little as possible in the buffer

\paragraph{The BPMN Extension is based on the Message element}
- if I want to talk about related work that goes another way
- why are the policies a parameter of the message and not the catch event element?

\paragraph{reuse existing technology}
-> we assume that a cep is present and that basic features of event queries can be used
- if not present, then only very basic buffer functionality is available
- but we dont want to start designing another event processing layer with duplicated functionality
