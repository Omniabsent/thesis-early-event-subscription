%************************************************
\chapter{Related Work}\label{ch:relatedwork}
%************************************************
% start broad, become narrower, explain differences

The field of \ac{EdBPM} has developed from connecting \ac{CEP} and \ac{BPM} in an attempt to increase the quality and performance of business processes.
\acs{EdBPM} must be viewed from different perspectives.
In event-driven business process monitoring and business activity monitoring, the process engine is solely an event producer that publishes information to an event engine.
A considerable amount of research was undertaken to make use of CEP to detect process violations and execution performance. 
\ref{label}
BAM, process mining... \cite{janiesch:poc-eventdriven-bam}
%- automatic query derivation
In \cite{Krumeich2014EventDrivenBP}, the authors did a comprehensive survey of research efforts in the area event-driven business process management and conclude that... > they point out that most research is in this area

...

This work has investigated how a flexible event subscription handling can be incorporated in BPMN models.
It is therefor contributing towards a facilitation 

roughly: event-driven business process management (anything that correlates bpm and cep)

Research in the area of event-driven business process management has been undertaken for several years now.



\cite{Pufahl2017} makes use of events to control process executions and considers event subscription as an essential step in the process. However, the subscription time cannot be chosen flexibly.

\cite{mandal:2017} has been the starting point for this thesis, not only acknowledging the discrepancy between the event occurrence time in the real world and the event subscription time as interpreted from the BPMN standard, but also incorporating event subscription in the BPMN model.
Inspired by their work, this thesis revisits the topic from scratch
however: 

\cite{Cabanillas2014}: investigations on the connection between events and process models, leading to \cite{Baumgrass2016}: extension of bpmn for receiving events, process variables in the query

similar to chapter 4:
%Control the flow: How to safely compose streaming services into business processes
%Modeling and Execution of Event Stream Processing in Business Processes
\cite{appel2014modeling}: includes the streams and stream processing right into the process model
same: \cite{biornstad2006control}




(1)

> business activity monitoring
%N. Herzberg, A. Meyer, O. Khovalko, M.Weske, Improving business process intelligence with object state transition events, in: 32nd International Conference on Conceptual Modeling (ER), China, 2013
%Event-based monitoring of process execution violations



(2)
- edpc: baumgraß, re-eval decisions
- eda
\cite{decker2008instantiation}: Process Instantiation, talks a lot about subscription, but no flexible subscr time, nobuffer, only for instantiation

\cite{von2010integrating}: Integrating Complex Events for Collaborating and Dynamically Changing Business Processes; talk about un/subscription, extend ws-bpel with subscription, reporting, patterns, expression language, ..; scope, time of subscription and event buffering not discussed
also: \cite{juric2010wsdl}

\cite{Kunz2010}: acknowledge the lack of usability in CEP and address it by applying bpmn as graphical support for the definition of cep patterns
\cite{gabriel2016konzeptionelle}: generate queries from new graphical notation, anbieterunabhängige Modellierung von EdBPM


- use case implementations
\cite{estruch2012event}: event-driven manufacturing process; 


- correlating events to processes

- maybe: persisting events in cep platforms | or delayed delivery of events
> \cite{roth2010event}: Event data warehousing for complex event processing and \cite{buchmann2010event}:Event-Driven services: Integrating production, logistics and transportation
also \cite{li2007historic}: historic data in pub/sub
> but if i want to have that historic information, than I would have to issue a subscription some time before. I integrate everything in the process model

- event buffering for the application in a BPM environment may not be confused with internal buffering techniques used within event engines to perform load shedding, even out short term load peaks. 
%see https://books.google.de/books?id=MWCfC9OKaToC&pg=PA198&lpg=PA198&dq=event+processing+optimization+overlapping+windows&source=bl&ots=yfcNb0YGAg&sig=qrYgyfRV-bLesOcCyxpeBdCQTr0&hl=de&sa=X&ved=0ahUKEwit2czXytbVAhVCD5oKHWdYAGAQ6AEIPjAC#v=onepage&q=event%20processing%20optimization%20overlapping%20windows&f=false
% Stream Data Processing: A Quality of Service Perspective: Modeling von Sharma Chakravarthy,Jiang Qingchun, 9.3.6

