%************************************************
\chapter{BPMN extension for Flexible Event Subscription}\label{ch:flexibleeventsubscription}
%************************************************

Given the additional requirements and the shortcomings identified in the previous sections, the following two chapters present an extension to the BPMN event handling model.
At first, an extension to the Business Process Model and Notation (BPMN) is described, which aims at providing the Process Designer with more flexible Event Handling capabilities according to Requirement~\textit{R1}.
Afterwards, Chapter~\autoref{ch:automaticsubscription} clarifies the changes necessary to the event handling platform and the process engine to cover Requirements \textit{R2} and \textit{R3}.
\todo[inline]{While the presented concepts are kept as general as possible, they are grounded in an analysis of the Esper-based CEP Platform Unicorn and the open-source process Engine Camunda.}

To allow the flexible use of event subscription in BPMN models, a number of additional attributes must be added to the model. 
The extension should cover the Intermediate Catch Event, the Boundary Catch Event and the Receive Task, all three can be used to model the receiving of messages in BPMN.
\todo[inline]{Is it ok to reference the receiveTask as well? It might have different execution semantics as it doesn't reference the MessageReceiveEvent, but the Message directly.}
% receiveTask.messageRef; boundaryEvent.catchEvent.(message)eventDefinition.messageRef; intermediateCatchEvent.catchEvent.(message)eventDefinition.messageRef
To cover all three elements, the extension will be attached to the BPMN type \textit{tMessage}. \todo[inline]{a new <element> or how do we extend?}
According to the plain specification, the message type comprises an attribute \textit{name}, the name of the message, and \textit{itemRef}, the reference to a BPMN \textit{Item} definition. 
\todo[inline]{explain 'item'?}
In the following, the required additional attributes will be explained one after the other. The goal is to retain a stand-alone model that contains all information necessary to execute the subscription to the event source.

\section{Adding basic subscription information}

For a basic event subscription, an event~query, the platform address and optionally authorization information of the CEP~Platform is required. \todo[inline]{as explained in background}
\todo[inline]{We work with the following simplified process}
Assuming that there is one CEP platform for all events and processes, the latter two, i.~e. the basic platform information, are configured centrally for the current process execution environment. Hence they don't need to be specified for every given message element. 
\todo[inline]{that means that in the implementation we need additional configuration values -> implementation chapter}
The event query instead needs to be specified for every message and is added to the model as an extension attribute of type string, which should contain the full query as interpretable by the CEP platform.
A similar approach has been taken by X and Y, which aim at enriching BPMN models with subscription information without considering the time of subscription specifically.
\todo[inline]{They add fields ... to element ..., their primary goal is ...}

\todo[inline]{Where does this leave us? We now have the necessary information to issue the subscription when reaching the event element.}

\section{The time of event subscription modeled in BPMN}
- what example to reference?
- where to write it?
- what are the options?
- what are the implications? => when exactly is the subscription executed. That means we have events since that time.
- that gives the process designer the necessary tool to implement the examples. He can now pick from which time he wants to consider events.



\subsection{Subscription time as part of the BPMN Message element}

\subsection{The explicit subscription task}
- variation in notation from other paper

\section{Using Process Variables in Event Queries}

\todo[inline]{this could be added in the requirements and referenced}

% also see bpmn2 spec 10.3 pp. 233ff.

As shown in example~Z, it can be the case that the values of process variables shall be dynamically used in an event query.
Therefore, the name of the process variable should be part of the event query. At the time of subscription, the mentioned variable is dynamically replaced by its current value.
The exact notation for including process variables in event queries can vary depending on the applied query language as it may not interfere with any existing notation schemes.
For the use with the Esper query language, the following is suggested: The exact name of the variable has to be surrounded by curly brackets and preceded by a \textit{\#} character: \textit{\#\{VARIABLENAME\}}.
This notation is inspired by the usage of substitution parameters in SQL queries that are embedded in Esper. They take the form \textit{\$\{expression\}}.
\todo[inline]{reference esper docs 5.13.1. Joining SQL Query Results}

In the example, the process uses the latest GPS position for a certain truck. The truck is identified by its unique~ID which is part of the query: \textit{SELECT lat, lng from GPSUPDATE where truckid = \#\{truckid\} }.
\todo[inline]{missingref: dependent example}

The use of dynamic process variable values introduces an additional complexity: Depending on the time of event subscription, the value of the process variable might not yet be available.
- reference BPMN data elements: process INSTANCE variable
-> the variable value might only be available during instance execution
-> can we find an exact definition of this in the spec?
- during execution the variable data might or might not be available. Related Work: Francesca?
-> too complex, we need a simplification for this.
- what happens if the data is not available?








- there is a lot we can do with this feature and event queries.
- some of the buffer policies mentioned in XY can be covered as follows:
- but there are cases when more advanced buffer handling can be helpful. consume and shared buffer.


\section{Advanced Buffer Parameters}










Present an abstract framework for flexible event subscription.
> Including: Model <> Process Engine <> Buffer <> CEP
> How does event subscription currently affect the workflow?
> What should a workflow look like that allows early event subscription?
> What must be explicitly stated by the user? What should be done automatically in the background?
(1 page)

To fulfill requirements R1.1 and R1.2, additional information has to be included in the BPMN model. By default, a BPMN intermediate event does not have information on the time of subscription or the event query. 
The BPMN specification offers BPMN-X extensions to add custom properties or elements to a model.

To accomodate the required information, the following extension is proposed:
> The extension should apply to MessageIntermediateCatchEvent and MessageBoundaryEvent
> extend tMessage => tBufferedCEPMessage, so that the messageRef can be reused
> OR extension to messageEventDefinition: ExplicitSubscriptionMessageEventDefinition
> [subscriptionQuery, subscriptionTime, bufferPolicy]

A buffer shared across multiple instances or events is more complex than a simple single-event-buffer (that one does not require buffer policies). As soon as the requestEvent call can be executed multiple times for the same queryId, we need to specify the following aspects:

Buffer policies:
(widely based on [Ref paper Sankalita]) RetrievalPolicy, ConsumptionPolicy, LifetimePolicy
+ buffer maximum age (= combination of lifetime policies)

