%************************************************
\chapter{Automatic Subscription Handling}\label{ch:automaticsubscription}
%************************************************

After defining the functionality provided to the user of flexible event subscription, this chapter describes the changes necessary to the software infrastructure that is used for event-driven business process management.
The concept requires that all subscription and event handling is executed by the system itself, without further interaction by the user.
All necessary information for that purpose is provided by the BPMN model.

As described in REF, an event-driven process management setup primarily consists of ...
\todo[inline]{missingref}
Changes are necessary to both, the Event Processing Module and the Process Engine. This chapter attempts to keep the change descriptions general so they can be applied to any common process engine and event processing platform.
The first section describes the necessary extension to the event processing to support early subscription and event buffering.
The following \autoref{ch:extendedprocessengine} specifies the changes necessary to the behavior of the process engine as the connecting element between the BPMN model and the event processing platform.


\section{Buffered Event Processing}\label{ch:bufferedcep}
When reduced to the basics, a standard event processing platform works as follows: The user subscribes to events providing an event query and a notification-path. The platform responds with a unique identifier for that subscription.
Whenever an event occurs that matches the provided query, the platform issues a notification to the notification-path. Subscriptions can be deleted through their unique identifier.
These two operations, \textit{subscribe} and \textit{unsubscribe}, make the fundamental API of a CEP platform.
\todo[inline]{ref} 


\paragraph{Evaluation of common CEP platforms}
- looking at the new bpmn extension reveals that a different behavior is required: notifications need to be kept in a buffer until they are requested by an entity
- three event platforms were studied to check if the required functionality can be implemented natively: wso2, esper and ...
- wso2: ?
- esper: 
- ? : ?
- a window will achieve something similar, but not exactly the same
- moreover, it is desired that the same full-featured event queries as before can be used, no restrictions. 
- as functionality wasn't available in any of the three, it was decided to specify an extended api.

\todo[inline]{Maybe: we consider these two operations as given, because common event processing platforms have these in common, but there is not common event buffering concept yet. even though e.g. esper has something which goes in that direction: output clauses}

\paragraph{An API for buffered Event Processing}
The novel BPMN extension for flexible event subscription allows to issue a subscription for buffering well before the events ought to be delivered via the notification-path. The introduction of an event buffer as a separate entity between the varying list of notification-recipients and the event query makes an extension of the API necessary.

\todo[inline]{note that splitting this into four actions is not a new concept. actually thats how event processing platforms normally operate, only without creation of the buffer and without delivering the buffered events}
\todo[inline]{what kind of API should this be? REST? Java? none specifically, but the reader might want clarification when reading this.}

Firstly, the \textit{subscribe} operation has to be divided into two steps:

\begin{aenumerate}
	\item \textit{registerQuery(queryString[, bufferPolicies]): queryId}\newline
	The call registers an event query in the CEP platform and instantiates a buffer. Matching events will be held in the buffer according to the specified  policies. It returns a unique identifier to that new query registration and hence for the connected buffer. That identifier must be used to modify the query later.
	
	\textit{bufferPolicies} is an optional parameter which is provided as an object with four possible fields: \textit{LifetimePolicy}, \textit{ConsumptionPolicy}, \textit{SizePolicy}, \textit{OrderPolicy}. Refer to \autoref{ch:bpmnx:bufferpolicies} for a detailed specification of the semantics of the parameters. If \textit{bufferPolicies} is not or only partly specified, the system should fall back to the default values.
	
	\item \textit{subscribe(queryId, notificationPath): subscriptionId}\newline
	Initiates the delivery of notifications for a given queryId to a notification recipient.
	The recipient is specified through the \textit{notificationPath}, the full address of the entity that is supposed to receive the message.
	Notifications are delivered asynchronously as soon as they are available. If the buffer is not empty, a message will be sent right after the \textit{requestEvents} call.
	A similar operation, \textit{addNotificationRecipient}, is available in existing CEP platforms. The difference is in the delivery of the first buffered message: \textit{requestEvents} sends out the message from the buffer, \textit{addNotificationRecipient} will send out notifications only for future query output.
	\todo[inline]{be clearer about notificationPath. Specify in background and reference. also about addNotificationRecipient}
\end{aenumerate}\label{def:apiextension-subscribe}

\todo[inline]{add table with buffer policies, their possible values and the default val}

A similar situation holds for the un-subscribe operation: Traditionally, a subscription is canceled through a unique identifier that is obtained as a result of the subscribe operation. After cancellation, no more notifications are delivered, the query is removed from the system.
Given flexible event subscription, this operation must be split into two parts as well:

\begin{aenumerate}
	\setcounter{enumi}{2}
	\item \textit{unsubscribe(subscriptionId)}\newline
	Removes a notification-recipient for a given query-id. Note that the buffer and query instance remain intact, so that other recipients can still subscribe.
	\item \textit{removeQuery(queryId)}\newline
	Completely deletes the query and its buffer, so that no notifications are sent out any longer. 
\end{aenumerate}\label{def:apiextension-unsubscribe}

\noindent
All four methods must be available to execute a subscription on process deployment.
The query must be registered using \textit{registerQuery} before the process instance, i.~e. notification-path, is available. For each process instance, a subscription can be issued individually using \textit{subscribe} and thereafter, the notification-recipient can be removed with \textit{unsubscribe}.
The query and its buffer will remain active even after any single instance has terminated. When the process gets un-deployed, the query can be deleted using \textit{removeQuery}.
\autoref{ch:extendedprocessengine} describes the steps in detail.
\todo[inline]{show the steps with sample data from one of the examples}

- to achieve the default behavior, each two steps have to be executed just after the other.


\missingfigure{maybe a uml sequence diagram. Or other diagram format?}
\todo[inline]{maybe provide a swagger definition for this?}


\paragraph{Architectural options to implement the extended API}

That extension can either be implemented by adopting the CEP platform itself, by implementing a separate middleware between process engine and CEP platform, or by implementing a buffering module as part of the process engine.
Which of the three options suits best has to be evaluated for the given use-case and existing infrastructure. In some cases it might not be possible to adapt the code of process engine or event processing platform, which leaves a separate middleware as the only choice.
\todo[inline]{Generally, extending the event processing platform is advisable? Or put up brief pros and cons of each of the options}
\missingfigure{three options to implement the event buffering and api extension}

A reference implementation for an extended complex event processing platform is presented in \autoref{ch:implunicorn} at the example of the Esper-based CEP~platform \textit{Unicorn}. It also explains, why extending the event platform was the preferred choice in the given scenario.

\todo[inline]{the communication overhead must be considered: if many events are sent to the buffer, but they are rarely issued to a process instance, then it will make sense to place the buffer closer to the event engine w.r.t. the the involved comm overhead.
Though it must be noted, that if events occur rarely, but many process instances consume them, than there will be an additional overhead to send events from the buffer to the instance. in that scenario, it might be advantageous to implement the event buffering close to the process engine}

\paragraph{Performance Considerations}

- performance improvements through shared windows
- due to its performance optimizations, an extension of the cep itself would make perfect sense
- given the extended api, it is now possible to implement early event subscription from the process engine
\todo[inline]{write}


\section{Extended Process Engine Behavior}\label{ch:extendedprocessengine}
It is the task of the Business Process Engine to interpret and execute process models and connect to an event processing platform in event-driven setups.
From the three relevant parts, two have already been defined, the BPMN extension and the buffered event processing module.
Out of the box, a process engine like Camunda will ignore any proprietary BPMN extensions and the subscription to an event source must be especially implemented. An example for such an implementation is provided in \autoref{ch:assessment-implementation}.
One goal of this work is to automatize the handling of event subscriptions solely based on the information available through the extended BPMN model. Additional process elements should not be required.
This section will clarify, which operations need to be executed by the process engine to enable the automatic subscription handling.
\autoref{ch:implcamunda} demonstrates the implementation of automatic subscription handling at the example of Camunda.

\paragraph{Parsing additional information from the BPMN model}
It is required that the process engine is able to read the additional information from the BPMN extension~(see~\autoref{ch:bpmnx}) so that it is available during process deployment and execution.
This affects the BPMN message element, which can contain the additional attributes \textit{eventQuery}, \textit{subscriptionTime} and \textit{bufferPolicies}.
Secondly, the \textit{Explicit Subscription Task} has to be processed. It contains a reference to a Message entity within the same model element.
The process engine might have to be adopted to read all relevant data from the extended model.


\paragraph{Managing subscription and un-subscription}
As defined in the BPMN extension for flexible event subscription, the action of subscribing to an event source can happen at different times during process deployment and execution. The options and the implicit timing of subscription and un-subscription are specified in \autoref{ch:bpmnx:subscriptiontimes}.
The process engine must communicate with the process engine using the four calls \textit{registerQuery}, \textit{requestEvents}, \textit{unsubscribe} and \textit{deleteQuery}, that were presented in the previous chapter.
For each possible subscription time, the following briefly enumerates which operations must be executed when. 

\begin{description}
	\item[In every case:] 
		The return-value of \textit{registerQuery}, a unique identifier of that query, must be stored for the related BPMN \textit{Message}. The id is later necessary to execute the other three API methods.
		When an event element is reached, a call to \textit{requestEvents} must be issued. When the execution of that event element is finished, call \textit{unsubscribe}.
	\item[Subscr. on process deployment:]
		When a process gets deployed, the process engine must check if subscription information is in the model. For every \textit{Message} element that is set as \textit{subscribe on pr. deployment}, a call to \textit{registerQuery} must be issued as part of the deployment process. A call to \textit{deleteQuery} is executed when the process gets un-deployed for the same messages.
	\item[Subscr. on process instantiation:] 
		When a process gets instantiated, \textit{registerQuery} must be executed for each \textit{Message} that is set to \textit{subscribe on pr. instantiation}. \textit{deleteQuery} can be called when the last reachable event element for a \textit{Message} has finished executing or no connected event can be reached anymore. The deletion happens at the latest when the process instance terminates.
	\item[Subscr. through explicit subscription task:]
		If the control flow reaches a subscription task, the process engine executes \textit{registerQuery} for the referenced \textit{Message}. The execution of \textit{deleteQuery} follows the same rules as in the preceding case.
	\item[Subscr. when the event element is reached:]
		Once the event element is reached, \textit{registerQuery} must be executed for any \textit{Message} that is not covered by one of the prior cases. \textit{deleteQuery} must be called when the event element is finished.
\end{description}
\todo[inline]{be more precise about the time the calls should be executed (if possible). "reached"? "completed"? use the right bpmn words}


\todo[inline] {Do I want to write about extended validity checks? soundness? The question would be: What happens if the model erroneous w.r.t. to the bpmn extension?}
- handling subscription dependencies:
- when are process variables replaced by their actual values
- The use of dynamic process variable values introduces an additional complexity: Depending on the time of event subscription, the value of the process variable might not yet be available.
- reference BPMN data elements: process INSTANCE variable
-> the variable value might only be available during instance execution
-> can we find an exact definition of this in the spec?
- see BPMN2 spec pp.211+ : Process and Activity can have DataInput and DataOutput. DataInput can have an 'optional' attribute
- during execution the variable data might or might not be available. Related Work: Francesca?
-> too complex, we need a simplification for this.
- what happens if the data is not available?


\todo[inline]{Time of un-subscription also must be clarified in bpmnx}
\todo[inline]{did i cover all requirements?}
