%************************************************
\chapter{Automatic Subscription Handling}\label{ch:automaticsubscription}
%************************************************

\section{Buffered Event Handling}
Why do we need a buffer to allow early event subscription?

What is the desired functionality of the event buffer? What functionality (API) does it expose?
> this could be seen as an extension to the API that is exposed by a standard CEP Platform
> standard platform API: registerQuery(queryString, notificationRecipient) : queryId, deleteQuery(queryId)
> extended API: registerQuery(queryString): queryId, requestEvent(queryId, notificationRecipient), unsubscribe(queryId), deleteQuery(queryId)


- buffer scope


\section{Extended Process Engine Behaviour}
=> As a link between the BPMN model and the Buffered Event handling

\todo[inline]{there must be a "subscription-garbage-collection" for any events that cannot be reached anymore in the current process execution! e.g. two different events behind an xor-gateway. the garbage collection could be executed on every transition}


- handling subscription dependencies:
- The use of dynamic process variable values introduces an additional complexity: Depending on the time of event subscription, the value of the process variable might not yet be available.
- reference BPMN data elements: process INSTANCE variable
-> the variable value might only be available during instance execution
-> can we find an exact definition of this in the spec?
- see BPMN2 spec pp.211+ : Process and Activity can have DataInput and DataOutput. DataInput can have an 'optional' attribute
- during execution the variable data might or might not be available. Related Work: Francesca?
-> too complex, we need a simplification for this.
- what happens if the data is not available?