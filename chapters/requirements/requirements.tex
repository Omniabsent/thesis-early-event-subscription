%************************************************
\chapter{Problem Statement}\label{ch:problemstatement}
%************************************************

This section will further define the problem and derive formal requirements to event subscription mechanisms

\section{Motivating Examples}
- one example for independent. The subscription does not depend on a prior precess result, the subscription can be done even before process instantiation

- one example for a process that uses an intermediate event that depends (subscription-wise) on the result of a previous step in the process.

\section{Event Subscription Scenarios}
What is meant by Event Subscription Scenario and why are we going to work with well-defined scenarios?
When can events happen in relation to the process (instance) lifecycle? => timeline
When must subscription happen to catch these events?
Second dimension: Does the subscription depend on a prior process result?
> A matrix of Event Subscription Scenarios.

\section{Requirements Definition}
Derive formal requirements, define, make them measurable.

R1: Flexible Event Subscription Time:

R1.1: Explicitness
> for each event that is used in a business process, the time of subscription must be clearly defined in relation to the process execution lifecycle
> The definition is done in the process model
> explicit about the event platform to subscribe to
> ? What's the granularity?

R1.2: Flexibility
> The subscription time can be chosen at least from the following options: [see scenarios]
> Options are limited when the subscription depends on data from previous execution steps

R2: Automatic Subscription Handling

R2.1: Subscription
> The subscription to event sources is handled implicitely during process deployment and execution
> It is handled according to the modeled subscription time.

R2.2: Unsubscription
> The unsubscription from an event source is handled automatically as soon as a subscription becomes unnecessary.

R3: Event Buffering
> All events since the subscription time are available to the process.
> A single buffer entity can be accessed from different process elements, process instances and processes within the environment
> Buffer policies?

