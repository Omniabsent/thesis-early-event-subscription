%************************************************
\chapter{Conclusions}\label{ch:conclusion}
%************************************************



By implementing the flexible event subscription concept in Unicorn and Camunda it could be demonstrated that state-of-the-art BPM and CEP technology is flexible enough to handle 

\section{Discussion}

- discussion:
> it remains the problem, that epl knowledge is not available in process design, a problem that has been addressed in ...
> while the event buffering can now be controlled from the process model level, events must still be acquired by the event engine before (set up event types and make sure that the events are pushed to the engine)
> the concept tries to bridge the gap between event persistence/historic events and real-time event processing by arguing that event occurrences are kept for a limited time and can therefor still be treated as events. A contra argumentation might be that an event becomes a simple piece of information as soon as it is not instantaneously consumed. hence it should not be modeled as an event but simply as a data object, using a data store
> many design decisions have been begründet with an improved usability though no empirical basis was available. Reasoning was only based on literature and chats with a small group of fellow researchers

- it can be argued that at process design time you dont care about the subscription time itself, but about the maximum age you can accept for your events. this would connect the lifetimepolicy and the subscription time into one value. but make the processing more complex and more fuzzy

\section{Future Work}


- this work attempts to provide a standard for handling event subscription in bpm architectures. given the necessity and the repeated attempts to address to topic, it would be a reasonable next step to discuss available solutions and start providing a foundation that is accepted and reused across the bpm community. No matter if based on the solutions of this work or not.
> it must be evaluated if that's the way event buffering shall be used. > prove the value of flexible event subscription in an industry case study
- future work: event data from other events could be used as historic data to allow the access even before process deployment
